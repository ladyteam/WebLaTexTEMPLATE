% This is a simple sample document.  For more complicated documents take a look in the exercise tab. Note that everything that comes after a % symbol is treated as comment and ignored when the code is compiled.

\documentclass[journal=jacsat,manuscript=article]{achemso}

\usepackage[version=3]{mhchem} % Formula subscripts using \ce{}
\usepackage{balance}
\usepackage{booktabs}
\usepackage{amsmath}
\usepackage{rotating}
\newcommand*\mycommand[1]{\texttt{\emph{#1}}}
%% ---
\usepackage{hyperref}
\urlstyle{same}
\hypersetup{
    colorlinks=true,
    linkcolor=blue,
    filecolor=magenta,      
    urlcolor=cyan,
}
\usepackage{bm}
%% Highlighting
\usepackage{soul}
\usepackage[table,usenames,dvipsnames]{xcolor}
\newcommand{\er}[1]{{\textcolor{Mahogany}{ER: \sf #1}}} % Notes by ER
\newcommand{\ab}[1]{{\textcolor{blue}{AB: \sf #1}}} % Notes by AB
\newcommand{\comment}[1]{} % comment in block, hide text

%% Line numbering ---
\usepackage[pagewise]{lineno}
\def\vsmall{\fontsize{9pt}{9pt}\selectfont}
\renewcommand{\linenumberfont}{\normalfont\sffamily\vsmall\color{lightgray}}
\linenumbers
%% ---
%% Page header
\makeatletter
\def\vsmall{\fontsize{9pt}{9pt}\selectfont}
\renewcommand{\@oddhead}{{\vsmall\sf%
\hfill\textcolor{gray}{Draft version, \today}}}
\renewcommand{\@evenhead}{\@oddhead}
\makeatother 

\author{Evgenii M.~Roginskii}
\affiliation[Ioffe Institute]{Ioffe
Institute, Polytekhnicheskaya 26, 194021 St. Petersburg, Russia}
\email{e.roginskii@mail.ioffe.ru}

\title{Simple Sample} % Sets article title
%\date{\today} % Sets date for date compiled
\abbreviations{Raman, non-linear optics, DFT}
\keywords{keyword1, keywors2}

\begin{abstract}
  We provide an ab-initio computational procedure 
\end{abstract}



\begin{document} % All begin commands must be paired with an end command somewhere
\maketitle % creates title using information in preamble (title, author, date)

\section{Hello World!} % creates a section

\textbf{Hello World!} Today I am learning \LaTeX. %notice how the command will end at the first non-alphabet charecter such as the . after \LaTeX
\LaTeX{} is a great program for writing math. I can write in line math such as $a^2+b^2=c^2$ %$ tells LaTexX to compile as math
. I can also give equations their own space:
\begin{equation} % Creates an equation environment and is compiled as math
  \gamma^2+\theta^2=\omega^2
\end{equation}
If I do not leave any blank lines \LaTeX{} will continue  this text without making it into a new paragraph.  Notice how there was no indentation in the text after equation (1).
Also notice how even though I hit enter after that sentence and here $\downarrow$
\LaTeX{} formats the sentence without any break.  Also   look  how      it   doesn't     matter          how    many  spaces     I put     between       my    words.

For a new essay I can leave a blank space in my code.

\end{document} % This is the end of the document